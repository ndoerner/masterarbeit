\documentclass[a4paper, 11 pt, fleqn]{article}

% Layout und Allgemeines
\usepackage[left = 2.5 cm, right = 2.5 cm, top = 2.5 cm, bottom = 2.5 cm ]{geometry} % Seitenränder
\usepackage[utf8]{inputenc} % Encoding
\usepackage{microtype} % verbesserter Randausgleich
\usepackage[onehalfspacing]{setspace} % Zeilenabstand 1,5
% Bibliographie
\usepackage[authoryear]{natbib}
% Sidenotes
\usepackage{sidenotes}
\usepackage{tikz}
\usetikzlibrary{calc}
\usepackage{makecell}
\usepackage{amsmath}
\usepackage{amssymb}
\usepackage{dsfont}
\usepackage{mathrsfs} % rsfs fonts in math mode
\usepackage{float}
\usepackage{multirow}
% Grafiken/Abbildungen
\usepackage{graphicx} 
\usepackage{subcaption}
\usepackage{pdflscape}
\usepackage{hyperref}

% Bunter Text
\usepackage{xcolor}
\graphicspath{{images/}}
\DeclareGraphicsExtensions{.pdf,.png,.jpg}
\setcitestyle{aysep={}, yysep={;}}
\bibliographystyle{apalike}

\DeclareMathOperator*{\argmax}{arg\,max} % argmax
\DeclareMathOperator*{\argmin}{arg\,min} % argmin

\tikzset{
  solid node/.style={circle,draw,inner sep=1.2,fill=black},
  hollow node/.style={circle,draw,inner sep=1.2},
}

\begin{document}
\begin{titlepage}
	\noindent
	Otto-Friedrich-Universitätt Bamberg\\
	Fakult\"at f\"ur Sozial- und Wirtschaftswissenschaften\\
	Lehrstuhl f\"ur Volkswirtschaftslehre, insb. Finanzwissenschaft\\
	Prof. Dr. Florian Herold
	
	\begin{center}
		\vspace{2 cm}
		\Large{
			\textbf{
		Consumer Responses to Behaviour-Based Price Discrimination
		}}\\ 
		\vspace{0.5 cm}
		\normalsize{Datum der Abgabe: 21.04.2023}\\
		\vspace{1 cm}
		\includegraphics[scale=0.175]{graphics/logo.png}\\
		\vspace{1 cm}	
		\textbf{Masterarbeit \\
		im Studiengang \\
		European Economic Studies (EES) \\
		an der Fakultät Sozial- und Wirtschaftswissenschaften \\
		der Otto-Friedrich-Universität Bamberg}\\
	\end{center}
	
	\vspace{5cm}
	\flushright{
		Niklas Sebastian Dörner\\
		An der Weberei 3, 96047 Bamberg\\
		Matrikelnr. 1859018\\
		niklas-sebastian.doerner@stud.uni-bamberg.de\\
	}			
\end{titlepage}

% Verzeichnisse
\pagenumbering{Roman}
\newpage
\addcontentsline{toc}{section}{Table of Contents}
\tableofcontents
\listoffigures
\newpage

% \listoftables

\newpage

\pagenumbering{arabic}

% Endlich kommt mal Inhalt
\begin{abstract}
	The abstract
\end{abstract}
%
\section{Introduction} \label{sec:intro}
%
\citet{Choe.2018} is chosen due to its significant contribution to the literature. The model they develop is dynamic and
was innovative in the sense that endogenously created asymmetric information leads to multiple and asymmetric equilibria with
and without endogenous product choice even when firms are symmetric ex ante. The asymmetry in equlibrium depends on the foward-looking
tendency of firms (positive discount factor for the future). \textcolor{red}{Siehe Content-Reiter Citavi}

\subsection{Price Differentiation} \label{ssec:price-diff}
Types of price discrimination
%
\subsection{Spatial Competition}
Mostly Hotelling (incl. impossibility result $\to$ relevance for Choe) and maybe also the circle (see Belleflamme)
%
\section{\citet{Choe.2018}: BBPD and Spatial Competition in a Dynamic Model} \label{sec:choe-2018-intro}
%
This section provides an introduction to the model presented in \citet{Choe.2018}. The present a model that combines
behaviour-based price discrimination with spatial competition. A brief overview of the model is followed by a thorough
exposition below. The model is dynamic and comprises two periods and two firms.
In the first period, firms choose their location as they do in the standard Hotelling model with quadratic transportation costs
but also track all consumers that decide to purchase from them in this period. This creates informational asymmetries as
firms can only track customers that purchased from \textit{them}. Once chosen, locations are fixed for the entirety of the
game. Note that the competiton for location is also a competition for information. In the baseline model, this thus concludes the
spatial competition between firms. Customer's location in the Hotelling linear city over [0,\,1] is the only source of heterogeneity
in the set of consumers.\footnote{This is sufficient as a source of heterogeneity as variation in willingness to pay based on different
transportation costs is the only property required by the model.} In the second period, firms can exploit their private information by
discriminating against consumers based on their decisions from the first period:~firms offer `personalised pricing' (p.~5672) to
\textit{each} consumer who purchased from that firm in the first period. Firms also set a uniform `poaching' price (p.~5672) in
order to attract consumers who previously purchased from their competitor. Firms simultaneously choose the poaching price first,
after which they set the individualised prices for their existing customers. This price structure hinges on the ability of firms
to track their customers in the first period and endogenously create private information. Consumers then observe three prices
prior to their purchase decision in the second period: their personalised price and the two poaching prices; they only care about
the lowest price absent any loyalty or fairness concerns. The authors assume that the inclusion of ``bevioral elements to consumer
choice is likely to change [the] results substantially.'' (p.~5672) The model is dynamic since firms have multiple periods to act,
multiple decisions to make and interact with one another. Accordingly, firms intertemporally maximise their total profits over both
periods. The authors thus endow firms (and consumers) with positive discount factors. The discounting of future profits by firms
ensures the existance of asymmetric equilibria; if firms have no intertemporal prefence for second-period profits, equilibria are
symmetric irrespecive of consumers' discount factors.
%
\subsection{Baseline Model} \label{sec:choe-baseline} % based on Section 2 of \citet{Choe.2018}
The firm comprises two firms that compete in a Hotelling linear city with quadratic costs for location and consumers in a two-stage game
with perods $\tau = 1,\,2$. Consumers are uniformly distributed on [0,\,1] where consumer $x$ is simply located at $x$.
Consumers incur quadratic transportation costs. Firms have full knowledge of this setup. Furthermore, firms are assumed to have constant
marginal cost of production; this is~--~as is common in this strain of the literature~--~normalised to 0. Over the course of the game,
firms gather information and engage in price discrimination against consumers. Consumers are price takers and simply act as a source of
demand without having any notable actions to chose from.
%
\subsubsection{Firms}
\textcolor{red}{Grafiken erstellen, um Spiel aus Sicht der Firma anschaulicher darzustellen?}
Firms are indexed by $i = A,\,B$ and act as duopolists. Firms compete for location for $\tau=1$. Since firms can only gather relevant
information in the first period, the location game is at the same time a contest for (private) information. The outcome of this game
determines the distribution of endogenously created private information among the dupolists and their respective ability to engage in
price discrimination. Firms set a unform price $P^1_i$ in the first period and track their customers. For each consumer, a firm knows
whether it was a customer in the first period. \citet[p. 5672]{Choe.2018} describe the information gain from tracking as follows:
let $\mathscr{A}$ denote the set of consumers that are a customer of firm $A$ in $\tau=1$. After $\tau=1$, firm $A$ knows
$\forall x \in [0,\,1]$ whether $x\in \mathscr{A}$ and also the location~$x$ of consumer~$x$ if that consumer was indeed a customer
of firm~$A$ in the first period. This similarly holds with customer set~$\mathscr{B}$ and customers~$y$ for firm~$B$.
The model structure assumed by \citet{Choe.2018} in conjunction with the tracking mechanism
implies that both firms know the location of the indifferent consumer \textcolor{red}{$\hat{x}$ benennen ja/nein?} (marginally above (below) the rightmost (leftmost) consumer in their
set of first-period customers, depending on whether the firm served the customers to left (right) of the indifferent consumer) and can, with
certainty, assign non-customers their first-period supplier since there are only two firms and the world is linear. If the model were to include
$n > 3$ competing firms, setting an efficient poaching price would likely be a much more involved process for any given firm. \\
% Period 2
The pricing strategies available to firm $i$ in period $\tau=2$ are determined by the informational asymmetries that result from the outcome
of the location game. \citet[p. 5672]{Choe.2018} let firms set two different types of prices in period $\tau=2$. Based on the private
information obatained in $\tau=1$, firms engage in price discrimination in two disctinct ways. They\footnote{For brevity I assume the perspective
of firm $A$ in the following description of price setting.} discriminate \textit{between} first-period customers (all $x\in\mathscr{A}$)
and consumers that chose the competitor (all $x\notin\mathscr{A}$, i.e. all $x\in\mathscr{B}$ since the authors assume a duopoly) but also
\textit{within} $\mathscr{A}$. Firms set a uniform poaching price, e.g. $P_A(\mathscr{B})$, aimed at all consumers who chose their competitor
in the first period and an individualised price, e.g. $P_A(x)$. The poaching price is designed to incentivise consumers to switch to another
firm in $\tau=2$ while the individualised pricing schedule is designed to appropriate the highest possible surplus based on the information
obtained in the first period.
However, each consumer chooses among their personsalised price and the two poaching prices. Note that the uniform poaching price can be viewed
as primarily a function of set membership, while the individualised price offered to a given customer $x\in\mathscr{A}$ is a function of their
specific location along the [0,\,1]-line \textcolor{red}{Mit oder ohne Bindestrich?}.
Firms set their poaching price first, followed by the individualised prices (p. 5672). \textcolor{orange}{ibid? Autor ja/nein? Nur Seitenzahl?}
Discriminating between the two major groups of consumers is \textcolor{red}{first/second/third; nachschauen was Choe et al. schreiben, aber auch mal neben Definitionen legen}
price discrimination. Price discrimination against the consumers in $\mathscr{A}$ (or $\mathscr{B}$, respectively) is first-degree price discrimination. \\
% Discount factor
\citet[p. 5672]{Choe.2018} endow firms with a positive discount factor $\delta_f \in [0,\,1]$. The discount factor is identical for both
firms but may differ from the discount factor of the consumer population, $\delta_c$. The model produces asymmetric equilibria as long as
$\delta_f > 0$ (p. 5673). Firms thus face an intertemporal decision problem to maximise their total discounted profit
$\Pi_i = \pi^1_i + \delta_f\pi^2_i$ and compete for location (information) based on their pricing regimes.
%

%
\subsubsection{Consumers}
The model assumes a single set of homogenous consumers. Consumers are indexed by their location $x$ in Hotellings linear city.
All consumers are within the interval $[0,\,1]$. Consumers incur transportation costs as a quadratic function of distance travelled
to their preferred retailer. All consumers have the same valuation $v$ for the good sold by the firms and only differ in their location
within Hotellings linear city. Furthermore, all consumers have an identical discount factor, $\delta_c$, which may differ from $\delta_f$,
the discount factor of the duopolists; while \citet{Fudenberg.2000} assume $\delta_c = \delta_f = \delta$, \citet{Choe.2018} assume the
general case of $\delta_c \neq \delta_f$ (p. 5673). For $\delta_c$ consumers are completely myopic and disregard the utility derived
from the second-period purchase while $\delta_c = 1$ implies that consumers value both instances of purchasing the good equally.
Consumers are price takers and are de facto only a source of demand for firms; they have no meaningful action set. The only variation
among consumers results from their location on the $[0,\,1]$ interval and by extension the different transport costs that are incurred
by different consumers. Household in a given period is defined as $v - P_i(x)-t(x-l)^2$ when buying from firm~$i$ (p. 5672). $v$ is a
consumers utility from consumption, $P_i(x)$ is the price set by firm~$i = A,\,B$ for consumer $x$. That firm is located at $l$, implying distance
travelled is $|x-l|$. Recall that firms set a uniform price in the first period, a uniform poaching price in the second period but also engage
in personalised pricing in period where possible. Thus, consumers effectively encounter five different prices over the course of the game. 
While never explicitly stated, consumers seem to maximise expected utility 
\begin{align} \label{eq:consumer-expected-utility}
	v - P^1_i(x)-t^1(x-l_i)^2 + \delta_c [v - P^2_j(x)-t^2(x-l_j)^2]
\end{align}
which is simply the discounted sum of the individual utilities. Here, the superscripts refer to the period in which the price is paid and the
transport costs are incurred, respectively. Making this disctinction for prices is obvious. However, since consumers may switch from one firm
to another when making their purchase decision in $\tau = 2$, consumers may travel some other distance relative to $\tau = 1$ and thus incur
the corresponding transportation costs. Since customers can remain loyal to their first-period choice, $i = j$ is possible.
\textcolor{red}{In Section~\ref{sec:choe-extensions} I consider extensions that introduce heterogeneity to the set of consumers, such as departing
from the assupmtion of a single discount factor $\delta_c$.}
%
\subsubsection{Equilibria and Theoretical Results} \label{ssec:choe-results}
\textcolor{red}{Wie detailliert soll auf die Ergebnisse aus \citet{Choe.2018} eingegangen werden? Behandlung der Herleitungen/Beweise aus dem Anhang?}
Fälle
\begin{itemize}
	\item Section 3: Exogenously fixed locations
	\item Section 4: Endogenous location choice
\end{itemize}
\textcolor{red}{Ein paar einleitende Sätze} \\
\subsubsection*{Exogenously Fixed Locations}
First, the authors solve the model by assuming exogenously fixed locations (p. 5673). They assume maximal differentiation, arguing that they
follow most other studies which treat behaviour-based price discrimination \textcolor{red}{Machen zB FT2000 das auch? Falls ja, wieso? Was, wenn man abweicht? Zb Hotelling ohne simultane Wahl, also Treffen in der Mitte?};
firm $A$ is assumed to be located at $0$, while firm $B$ is assumed to be located at $1$. Let the marginal consumer who is indifferent between
firm $A$ and $B$ \textit{in period $\tau = 1$} be consumer $z \in [0,\,1]$. Following from the location restrictions imposed on the firms, firm $A$ serves all
consumers $x \leq z$ and firm $B$ serves all consumers $y \geq z$, respectively, in the first period. \textcolor{red}{The position of the marginal
consumer follows from the outcome of first-period Bertrand game and by extension which of the two equilibria of the game materialises.}
\textcolor{cyan}{Ja, es sind nur zwei, da je ein Fall aus Lemma 1 zu einem Fall aus Lemma 2 korrespondiert.}
By exogenously fixing the position of both firms, the Bertrand game is the only relevant intercation in the first period.
It determines the position of the marginal consumer. The authors derive two asymmetric (albeit very similar) equilibria, one for each case, i.e. $(i) z < 0.5$
and $(ii) z > 0.5$. The equilibria are characterised in Lemma~1 (equilibria of the Bertrand game in $\tau = 1, $p. 5673) and Lemma~2
(equilibrium prices of the discrimination game as set in $\tau = 2$, p. 5675). The lemmatas refer to the original article. The full derivation
is in the appendix of the orginal article (p. 5682). The proof runs as follows: firms begin by determining their pricing strategies for $\tau = 2$.
As described in section~\ref{ssec:choe-firms} these prices are a function of consumer's set membership. The price set by an individual firm depends
on the outcome of the Bertrand game (i.e. $z \lessgtr 0.5$) but the pricing regime implemented by the firm that secured the larger market
share is similar in either case. Since locations are set exogenously, there is no other aspect to the determination of the location of $z$
in this case. Next, the specific location of the marginal consumer $z$ within the relevant interval is determined~--~in case~$(i)$,
$z$ is located somewhere in $[0,\,0.5)$, for example. The authors maximise firms' profits and exploit the first-order conditions for
profit maximisation to obtain their expressions for $z$, the reaction functions, and the equilibrium prices.
\textcolor{red}{Vermutlich sollte ich die Preise/Ausdrücke hier hinschreiben, um später darauf Bezug zu nehmen und um der Leserin einen Bezugspunkt
zu geben? Vor allem, da ja auch hierunter darauf Bezug genommen wird}\textcolor{cyan}{$\to$ wenn hier schon 3x Bezug steht, dann erübrigt sich die Frage wohl}\\
% Discussion of equilibria
%
The characterisation of the equlibria is described by \citet[p. 5682]{Choe.2018} as follows: \textcolor{red}{Beschreibung aus appendix und section 3 kombinieren}
For $z \leq 0.5$ firm $B$ will have secured the majority of the market in period $\tau = 1$. Both firms engage in first-degree price
discrimination against consumers within their respective customer sets \textcolor{red}{wie viel Konsumentenrente wird abgeschöpft? Falls alles, kann man dann noch von first-degree price discrimination sprechen?}
and set a poaching price for all other consumers. The firm which secured less of the market in $\tau = 1$, i.e. firm $A$ in this case,
sets a positive poaching price. Firm $B$ sets a price of $0$ for consumers in $[0,\,\frac{1+2z}{4}]$ since consumers to the left of $\frac{1+2z}{4} \leq 0.5 \,\forall z \leq 0.5$
find travelling to $B$ prohibitively expensive after observing the pricing schedule of firm $A$ as described by~\eqref{FEHLT NOCH} \textcolor{red}{FEHLT NOCH}.
This price of $0$ comprises both the uniform poaching price offered by firm $B$ and the personalised pricing offered to some $y \in \mathscr{B}$.
Consumers to the right of this threshold remain customers of firm $B$ and are subject to personalised pricing (captured by the inclusion of their
specific location in the price formation process) \textcolor{red}{wenn man sich den Beweis zu Lemma 1 anschaut ist der nicht immer > 0 oder?}
\textcolor{green}{Wie genau sieht eigentlich $\tilde{y}$ aus? Das muss auch noch als Schwellwert rein}
\\
\citet[p. 5674]{Choe.2018} argue that the firm, e.g. firm $B$, which was able to secure a larger share of the market in the first period also has more to gain from
a shift of the marginal consumer's position away from the center of the linear city, relative to what the competitor, e.g. firm $A$, loses if
such a shift were to occur. \textcolor{red}{Is there a convergence to the extremes? The authors do not mention such a dynamic in their discussion on p. 5674 $\to$ vllt mal die Ableitungen untersuchen? In der Ableitung des Gesamtgewinns taucht aber kein z auf (p.5683)} 
In the case of such a shift ($z' < z < 0.5$), the market share of $B$ in $\tau = 1$ increases relative to the majority it already controls.
Such a shift requires a reduction in $P^1_B$ as $B$ must offset the higher transportation costs for those located in $[z',\,z)$. This more aggressive
pricing affects all consumers previously already in~$\mathscr{B}$. As consumers maximise discounted intertemporal utility, firm $B$ is able to
charge higher personlised prices and thus has a strong incentive to aggressively price its product in $\tau = 1$. 
Conversely, firm $A$ can increase its poaching price in the second period relative to the situation prior to $z\to z'$, lowering the incentive 
to undercut the aggressive pricing of firm $B$, and supporting the asymmetric equlibria of the game (p. 5674). \\
Hence, the firm with the larger market share is incentivised to price more aggressively in the first period with the goal of securing a larger
market share and gain an information advantage over its competitor. While they will lose customers to poaching in the second period,
the ability to engage in personalised pricing allows firms to offset these losses in market share in terms of profits (p. 5680).
There is only one-way switching in equilibrium (p. 5674).
%
Furthermore, it can now be easily verified that the existence of \textit{asymmetric} equilibria does indeed
require that firms are forward-looking. Setting $\delta_f = 0$ in \eqref{hier label einfügen, Bezug auf Ergebnis von Lemma 1 } \textcolor{red}{FEHLT NOCH} implies $z = 0.5\, \forall\, \delta_c$. Setting $\delta_c = 0$ does not influence the
existence of asymmetric equilibria. A symmetric outcome of the location game in turn indicates that the information obtained by firms is qualitatively
identical. While firms track different consumers in $\mathscr{A}_{\delta_f = 0} = \{x | 0 \leq x \leq 0.5\}$ and $\mathscr{B}_{\delta_f = 0} = \{y | 0.5 \leq y \leq 1\}$, still under the
assumption that firms $A,B$ is located at $0$ and $1$ respectively, both obtain the same information in the sense that both serve half the consumer 
population and their respective customer bases are symmetric relazstive to the marginal consumer at $z = 0.5$ and for every consumer that frequents one
of the firms in $\tau = 1$, there exists a customer of the other firm that incurs identical transportation costs.
For given values of $\delta_c, \delta_f$ the prices set by firms in period $\tau = 1$ are simply linear functions of $t$,
the (constant) scaling factor used to compute transportation costs.
\subsection{Related Literature (tbd)}
\begin{itemize}
	\item p. 5672: ``we abstract away from these issues mainly because our aim is to clearly understand how changes in informational assumptions lead to different equilibria
	in otherwise the same model as Fudenberg and Tirole (2000)''
	\item die Autoren nennen 3 Referenzmodelle, mit denen sie sich vergleichen (ua FT2000), ggf gibt es noch andere; die im Artikel genannten sind relativ alt
	\item Ähnlichkeiten zu Armstrong/Vickers
	\item Ähnlichkeiten zu add-on market: es scheint, als ließen sich die Konsumenten aus Choe mit variierenden Diskontfaktoren in die Agenten aus dem anderen Paper umbauen. Myopes wären dann sogar (fast?) identisch, da sie die zweite Periode (äqivalent zum Add-on price) ignorieren (nicht kennen); einziger Unterschied sind Transportkosten?
\end{itemize}
%
\section{\citet{Armstrong.2019}: Discriminating Against Captive Consumers}
%
\subsection{Captive Consumers~--~Consumer Myopia}
%
\section{\textcolor{red}{Zwischenfazit: What we have learned so far, where we stand now and where we are going next}}
Preliminary conclusion: Model Components oder so ähnlich
%
\section{Extensions} \label{sec:choe-extensions}
%
\subsection{Intersecting \citet{Choe.2018} and \citet{Armstrong.2019}}
%
\subsection{Consumer Responses: Signalling and Search}
%
\subsubsection{Signalling}
%
\subsubsection{Search}
%
\newpage
\pagenumbering{Roman}
% \setcounter{page}{3}
\bibliography{bibliography}

\newpage
\section*{Selbständigkeitserklärung}
Ich erkläre hiermit gem. \text{\S}6 Abs. 6 APO SoWi, dass ich die vorstehende Masterarbeit selbstständig verfasst, keine anderen
als die angebenen Quellen und Hilfsmittel benutzt und die wörtlich oder inhaltlich übernommenen Stellen als solche kenntlich
gmacht habe. Des Weiteren erkläre ich, dass die digitale Fassung der gedruckten Asufertigung der Masterarbeit ausnahmslos
in Inhalt und Wortlaut entspricht und zur Kenntnis genommen wurde, dass diese digitale Fassung, einer durch Software
unterstützten, anonymisierten Prüfung auf Plagiate unterzogen werden kann. \\
\vspace*{1 cm}
\\
\noindent\begin{tabular}{ll}
	\makebox[2.5in]{\hrulefill} & \makebox[2.5in]{\hrulefill}\\
	Niklas D\"orner & Ort, Datum\\[8ex]% adds space between the two sets of signatures
\end{tabular}
%
\end{document}