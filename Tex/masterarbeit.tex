\documentclass[a4paper, 11 pt, fleqn]{article}

% Layout und Allgemeines
\usepackage[left = 2.5 cm, right = 2.5 cm, top = 2.5 cm, bottom = 2.5 cm ]{geometry} % Seitenränder
\usepackage[utf8]{inputenc} % Encoding
\usepackage{microtype} % verbesserter Randausgleich
\usepackage[onehalfspacing]{setspace} % Zeilenabstand 1,5
% Bibliographie
\usepackage[authoryear]{natbib}
% Sidenotes
\usepackage{sidenotes}
\usepackage{tikz}
\usetikzlibrary{calc}
\usepackage{makecell}
\usepackage{amsmath}
\usepackage{amssymb}
\usepackage{dsfont}
\usepackage{mathrsfs} % rsfs fonts in math mode
\usepackage{float}
\usepackage{multirow}
% Grafiken/Abbildungen
\usepackage{graphicx} 
\usepackage{subcaption}
\usepackage{pdflscape}
\usepackage{hyperref}

% Bunter Text
\usepackage{xcolor}
\graphicspath{{images/}}
\DeclareGraphicsExtensions{.pdf,.png,.jpg}
\setcitestyle{aysep={}, yysep={;}}
\bibliographystyle{apalike}

\DeclareMathOperator*{\argmax}{arg\,max} % argmax
\DeclareMathOperator*{\argmin}{arg\,min} % argmin

\tikzset{
  solid node/.style={circle,draw,inner sep=1.2,fill=black},
  hollow node/.style={circle,draw,inner sep=1.2},
}

\begin{document}
\begin{titlepage}
	\noindent
	Otto-Friedrich-Universitätt Bamberg\\
	Fakult\"at f\"ur Sozial- und Wirtschaftswissenschaften\\
	Lehrstuhl f\"ur Volkswirtschaftslehre, insb. Finanzwissenschaft\\
	Prof. Dr. Florian Herold
	
	\begin{center}
		\vspace{2 cm}
		\Large{
			\textbf{
		Consumer Responses to Behaviour-Based Price Discrimination
		}}\\ 
		\vspace{0.5 cm}
		\normalsize{Datum der Abgabe: 21.04.2023}\\
		\vspace{1 cm}
		\includegraphics[scale=0.175]{graphics/logo.png}\\
		\vspace{1 cm}	
		\textbf{Masterarbeit \\
		im Studiengang \\
		European Economic Studies (EES) \\
		an der Fakultät Sozial- und Wirtschaftswissenschaften \\
		der Otto-Friedrich-Universität Bamberg}\\
	\end{center}
	
	\vspace{5cm}
	\flushright{
		Niklas Sebastian Dörner\\
		An der Weberei 3, 96047 Bamberg\\
		Matrikelnr. 1859018\\
		niklas-sebastian.doerner@stud.uni-bamberg.de\\
	}			
\end{titlepage}

% Verzeichnisse
\pagenumbering{Roman}
\newpage
\addcontentsline{toc}{section}{Table of Contents}
\tableofcontents
\listoffigures
\newpage

% \listoftables

\newpage

\pagenumbering{arabic}

% Endlich kommt mal Inhalt
\begin{abstract}
	The abstract
\end{abstract}
%
\section{Introduction} \label{sec:intro}
%
\citet{Choe.2018} is chosen due to its significant contribution to the literature. The model they develop is dynamic and
was innovative in the sense that endogenously created asymmetric information leads to multiple and asymmetric equilibria with
and without endogenous product choice even when firms are symmetric ex ante. The asymmetry in equlibrium depends on the foward-looking
tendency of firms (positive discount factor for the future). \textcolor{red}{Siehe Content-Reiter Citavi}

\subsection{Price Differentiation} \label{ssec:price-diff}
Types of price discrimination
%
\subsection{Spatial Competition}
Mostly Hotelling (incl. impossibility result $\to$ relevance for Choe) and maybe also the circle (see Belleflamme)
%
\section{\citet{Choe.2018}: BBPD and Spatial Competition in a Dynamic Context} \label{sec:choe-2018-intro}
%
This section provides an introduction to the model presented in \citet{Choe.2018}. The present a model that combines
behaviour-based price discrimination with spatial competition. A brief overview of the model is followed by a thorough
exposition below. The model is dynamic and comprises two periods and two firms.
In the first period, firms choose their location as they do in the standard Hotelling model with quadratic transportation costs
but also track all consumers that decide to purchase from them in this period. This creates informational asymmetries as
firms can only track customers that purchased from \textit{them}. Once chosen, locations are fixed for the entirety of the
game. \textcolor{red}{VERSCHIEBEN Note that the competiton for location is also a competition for information.} In the baseline model, this concludes the spatial competition between firms. Customer's location in the Hotelling linear
city over [0,1] is the only source of heterogeneity in the set of consumers.\footnote{This is sufficient as a source of heterogeneity as variation in willingness to pay is the only thing required by the model.} In the second period, firms can exploit
their private information by discriminating against consumers based on their decisions from the first period:~firms offer
`personalised pricing' (p.~5672) to \textit{each} consumer who purchased from that firm in the first period. Firms also set
a uniform `poaching' price (p.~5672) in order to attract consumers who previously purchased from their competitor.
Firms simultaneously choose the poaching price first, after which they set the individualised prices for their existing customers.
This price structure hinges on the ability of firms to track their customers in the first period and endogenously create private
information. Consumers then observe three prices prior to their purchase decision in the second period: their personalised price
and the two poaching prices; they only care about the lowest price absent any loyalty or fairness concerns. The authors assume that
the inclusion of ``bevioral elements to consumer choice is likely to change [the] results substantially.'' (p.~5672)
The model is dynamic since firms have multiple periods to act, multiple decisions to make and interact with one another.
Accordingly, firms intertemporally maximise their total profits over both profits. The authors thus endow firms (and consumers) with
positive discount factors. The discounting of future profits by firms ensures the existance of asymmetric equilibria; if firms have no
intertemporal prefence for second-period profits, equilibria are symmetric irrespecive of consumers' discount factors.


\subsection{Related Literature (tbd)}
\begin{itemize}
	\item p. 5672: ``we abstract away from these issues mainly because our aim is to clearly understand how changes in informational assumptions lead to different equilibria
	in otherwise the same model as Fudenberg and Tirole (2000)''
	\item die Autoren nennen 3 Referenzmodelle, mit denen sie sich vergleichen (ua FT2000), ggf gibt es noch andere; die im Artikel genannten sind relativ alt
\end{itemize}
%
\section{\citet{Armstrong.2019}: Discriminating Against Captive Consumers}
%
\subsection{Captive Consumers~--~Consumer Myopia}
%
\section{\textcolor{red}{Zwischenfazit: What we have learned so far, where we stand now and where we are going next}}
Preliminary conclusion: Model Components oder so ähnlich
%
\section{Extensions} \label{sec:extensions}
%
\subsection{Intersecting \citet{Choe.2018} and \citet{Armstrong.2019}}
%
\subsection{Consumer Responses: Signalling and Search}
%
\subsubsection{Signalling}
%
\subsubsection{Search}
%
\newpage
\pagenumbering{Roman}
% \setcounter{page}{3}
\bibliography{bibliography}

\newpage
\section*{Selbständigkeitserklärung}
Ich erkläre hiermit gem. \text{\S}6 Abs. 6 APO SoWi, dass ich die vorstehende Masterarbeit selbstständig verfasst, keine anderen
als die angebenen Quellen und Hilfsmittel benutzt und die wörtlich oder inhaltlich übernommenen Stellen als solche kenntlich
gmacht habe. Des Weiteren erkläre ich, dass die digitale Fassung der gedruckten Asufertigung der Masterarbeit ausnahmslos
in Inhalt und Wortlaut entspricht und zur Kenntnis genommen wurde, dass diese digitale Fassung, einer durch Software
unterstützten, anonymisierten Prüfung auf Plagiate unterzogen werden kann. \\
\vspace*{1 cm}
\\
\noindent\begin{tabular}{ll}
	\makebox[2.5in]{\hrulefill} & \makebox[2.5in]{\hrulefill}\\
	Niklas D\"orner & Ort, Datum\\[8ex]% adds space between the two sets of signatures
\end{tabular}
%
\end{document}